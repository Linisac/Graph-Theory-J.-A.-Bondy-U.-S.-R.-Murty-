%\setcounter{chapter}{1}
\chapter{Graphs}
%Section 1---------------------------------------------------------------------------------------------------------------
\section{Graphs and Their Representation}
\begin{enumerate}[1.]
%
\item[\thesection.1] There are ${n \choose 2}$ unordered pairs of distinct vertices $uv$.
%
\item[\thesection.2]
\begin{enumerate}[a)]
%%
\item There are $rs$ ordered pairs $(x, y)$ of vertices $x \in X, y \in Y$.
%%
\item Note that $rs = r(n - r) = -(r - n / 2)^2 + n^2 / 4$.
%%
\item When $n$ is an even number, $r = s = n / 2$ and $\bipartite{G}{X, Y}$ is a complete bipartite graph.
%%
\end{enumerate}
%
\item[\thesection.3]
\begin{enumerate}[a)]
%%
\item Consider the path $\path[n]$ with vertices $v_0, \etc, v_n$ in which $n \geq 1$ such that every pair of consecutive vertices in this listing are adjacent. With $V_E$ and $V_O$ denoting the sets of odd- and even-indexed vertices of $\path[n]$, respectively, we have $\path[n] = \bipartite{\path[n]}{V_E, V_O}$.
%%
\item Consider the cycle $\cycle[n]$ with vertices $v_0, \etc, v_n$ in which $n \geq 3$ such that every pair of consecutive vertices in this listing as well as $\sete{v_0, v_n}$ are adjacent. Let $V_E$ and $V_O$ be as defined above.
%%
\end{enumerate}
%
\item[\thesection.4] Since for every $v \in \vset(G)$, $\mindgr(G) \leq \dgr(v) \leq \maxdgr(G)$.
%
\item[\thesection.5]
\begin{description}
%%
\item[$k = 0$.] Empty graphs.
%%
\item[$k = 1$.] Graphs consisting of separate pairs of vertices adjacent to each other.
%%
\item[$k = 2$.] Graphs consisting of one or more cycles.
%%
\end{description}
%
\item[\thesection.6]
\begin{enumerate}[a)]
%%
\item Assume there are $n \geq 2$ people in this question. Model this group of people as a graph $G$ where each person is represented by a vertex and the friendship between two people by an edge connecting the two vertices for the two people. Suppose for a contradiction that no two people have the same number of friends or, in the graph $G$, each vertex $v$ has a different degree $\dgr(v)$. Since $0 \leq \dgr(v) \leq n - 1$ for all vertices $v \in \vset(G)$ and since $\vnum(G) = n$, we have that there are exactly $n$ different vertex degrees in $G$: $0, \etc, n - 1$. Let us say the one vertex of degree $0$ (i.e.~an isolated vertex) is $v_0$, and denote by $G'$ the graph obtained by deleting $v_0$, namely $G' \defas \subgraph{G}{\vset(G) \setminus \sete{v_0}}$. Then $G'$ is a graph with $n - 1$ vertices and $\maxdgr(G') = n - 1$, which is a contradiction because $\maxdgr(G') \leq n - 2$.
%%
\item (INCOMPLETE) The first is like a five-ray star ($\clique[5]$ with the outermost $5$ edges removed).
%%
\end{enumerate}
%
\item[\thesection.7]
\begin{enumerate}[a)]
%%
\item (SKIPPED).
%%
\item $\vnum(Q_n) = 2^n$, and
\[
\enum(Q_n) = 
\begin{cases}
1 & \mbox{if \mathmode{n = 1}} \cr
2\enum(Q_{n - 1}) + 2\vnum(Q_{n - 1}) & \mbox{otherwise}. \cr
\end{cases}
\]
Solving the recurrence equation by means of generating function, we have $\enum(Q_n) = n2^{n - 1}$.
%%
\item Tuples that differ in an odd or even number of coordinates are in different or the same parts (of a bipartite graph).
%%
\end{enumerate}
%
\end{enumerate}
%End of Section 1--------------------------------------------------------------------------------------------------------